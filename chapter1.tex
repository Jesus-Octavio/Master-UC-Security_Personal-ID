%%  -*-coding: utf-8;ispell-local-dictionary: "castellano8";-*-
\providecommand{\main}{..}      % UB:28.11.2020: needed for the
                                % bibfile which is local

\documentclass[main.tex]{subfiles}

\begin{document}

\chapter{La identidad personal como problema filosófico.}	

\section{La continuidad de la sustancia material.}

Algunos criterios sobre la persistencia de la identidad personal a través del tiempo se basan en el hecho de tener una existencia material continua.

En primer lugar, abordaremos estos criterios para objetos físicos e inanimados.

\textit{Teseo ha navegado por los mares del mundo durante un año. Al concluir su travesía, observa que su nave se ha deteriorado gravemente. Por ello, la sitúa en un dique seco para repararla. Pero la reparación, que le lleva un año completo, es más seria de lo que parecía a priori. Teseo ha sustituido todas las piezas del buque por otras exactamente iguales a las anteriores. Concluida la reparación, Teseo vuelve al mar. Pero, lo hace en compañía, pues, mientras reparaba su nave e iba desechando sus piezas, un rival las restauraba una a una y ha construido un buque exactamente igual al de Teseo con el que se lanza al mar.}

Esta paradoja, propuesta por Plutarco (s. I - s. II), plantea varias cuestiones: ¿cuál de las dos naves, la que dirige Teseo o la que dirige el rival, se corresponde con la nave antigua?, ¿ninguna, ambas, la que dirige Teseo, o la que dirige su rival?

Los interrogantes aumentan al tratar con personas. Quizá, el sentido común parezca indicar que los juicios de identidad se basan en lo que llamaremos \textit{criterio corporal}: el mismo cuerpo es la misma persona. Si bien, no podemos esperar mucho de este criterio. ¿Puede servir el cuerpo físico para dar cuenta de la problemática del veinteañero y el niño de la fotografía? En general, se reconoce que el cuerpo de las personas sirve como criterio de identidad personal, pero más bien lo hace como criterio de evidencia y no como criterio constitutivo. Es decir, que el cuerpo sea el mismo, normalmente, es indicio de que la persona es la misma. Si bien, como ya se ha dejado entrever, los criterios de identidad personal no tratan de establecer qué va a aceptarse como evidencia de la identidad personal sino en qué consiste la identidad personal. Las huellas dactilares parecen ser una evidencia de identidad personal, pero nadie zanjaría la cuestión alegando que las huellas dactilares son en sí la identidad de una persona. Continuaremos con el ejemplo de las huellas dactilares y las manos en adelante. 

Derek Parfit (1942-2017) plantea el siguiente experimento mental para sacar a la luz las intuiciones sobre la continuidad corporal.

\textit{He estado en Marte, pero sólo por el método clásico, un viaje espacial que dura varias horas mientras estás acomodado en un teletransportador. Esta máquina me envía hasta allí a la velocidad de la luz tras pulsar un botón verde. Como es habitual, estoy muy nervioso. ¿Funcionará? Recapitulo los consejos e instrucciones que me han indicado. Cuando apriete el botón verde, perderé la conciencia y, tiempo después, me despertaré con la sensación de que ha pasado sólo un instante. Sin embargo, habré estado inconsciente un par de horas. El escáner situado en la Tierra, destruirá mi cuerpo y mi cerebro a la vez que graba los estados de todas mis células. Acto seguido, transmitirá esta información al Replicador de Marte a la velocidad de la luz. Este Replicador creará, a partir de materia nueva, un cerebro y un cuerpo exactamente iguales a los míos. Aprieto el botón. Según lo previsto, pierdo la conciencia y parece que la recupero en seguida, pero en un cubículo diferente. He llegado a Marte. Examino mi nuevo cuerpo y no encuentro ningún cambio físico.}

Esta historia nos muestra que quizá la continuidad corporal puede no ser un criterio para juzgar la identidad de una persona a través del tiempo sino sólo una evidencia importante pero falible. Quizá dé a entender que se prefiere un criterio aristotélico, es decir, que la intuición sobre la identidad personal apunta a cierta estructura inteligible -no material- que, en este ejemplo, se correspondería con la información que se ha grabado y teletransportado. O bien, que el criterio de continuidad corporal ofrecido por el sentido común se revela, simplemente, como falso.


Si la identidad personal se halla constituida por la identidad del cuerpo, entonces la identidad personal no se distinguiría en lo esencial de la identidad de los objetos físicos no animados, que vendría dada por su continuidad espacio-temporal. Por ello, el criterio corporal no puede exigir la continuidad material estricta -de manera similar a la nave de Teseo, en todo ser vivo se destruyen células degradándolas hasta sus sillares más básicos y, a partir de ellos, se construyen nuevas células; el cuerpo que el veinteañero ve en la foto no se corresponde con el suyo actual- sino más bien que el cambio material tenga lugar de una determinada manera. Lo que el criterio corporal requiere para que las identidades de la persona $P_{2}$ en el instante $t_{2}$ y de la persona $P_{1}$ en el instante $t_{1}$ sean la misma no es que $P_{1}$ y $P_{2}$ sean materialmente idénticas sino que la materia que constituye a $P_{2}$ sea resultado de la que constituye a $P_{1}$ tras una serie de sustituciones de manera que sea correcto decir que el cuerpo de $P_{2}$ en $t_{2}$ es idéntico al cuerpo de $P_{1}$ en $t_{1}$. Por ello y por los procesos celulares, estaría justificado decir que el cuerpo, luego la identidad, del veinteañero es el mismo que el del niño de la foto.


Pero aún caben más ejemplos que desafían al criterio corporal y que los defensores de los criterios de la continuidad material dicen solventar de manera efectiva. Sidney Shoemaker (1931-) muestra el siguiente experimento mental:

\textit{Supongamos que la cirugía ha alcanzado unos niveles de sofisticación nunca antes imaginados. Supongamos que la técnica para operar tumores cerebrales consiste en extraer el cerebro del cráneo, mantenerlo vivo mientras dura la intervención y volver a colocarlo en su sitio restableciendo todas sus conexiones originales. Cierto día, en una clínica, operan a los señores Brown y Robinson mediante el procedimiento descrito. Pero, por despiste, han reinsertado el cerebro de Brown en el cuerpo de Robinson y el cerebro de Robinson en el cuerpo de Brown. Uno de esos hombres, el que tiene el cerebro de Robinson y el cuerpo de Brown, fallece inmediantamente. El hombre con el cerebro de Brown y el cuerpo de Robinson sobrevive. Llamémoslo Brownson. Al despertar, Brownson recupera la conciencia y se horroriza al verse en el espejo. No reconoce su rostro. Tampoco su timbre de voz. Exige que le llamen Brown, tiene los recuerdos de la vida de Brown y, desde luego, pretende que le lleven a casa de Brown con la familia de Brown, no a casa de Robinson con la famila de Roinson, a la que no reconoce.
}

Casi todos coincidiríamos en que Brownson es Brown. Algunos defensores de los criterios de continuidad material alegarían que lo que necesario para la identidad personal no es la identidad del cuerpo completo sino la identidad del cerebro -en tanto que es el órgano que parece controlar la memoria, el carácter, la personalidad-. Por tanto, el \textit{criterio cerebral} afirma que $P_{2}$ en $t_{2}$ será la misma persona que $P_{1}$ en $t_{1}$ únicamente en caso de que $P_{2}$ en $t_{2}$ tenga el mismo cerebro que $P_{1}$ en $t_{1}$.



Ahora bien, existen reflexiones en torno a estudios con pacientes comisutorizados  -casos de bisección cerebral que derivan en la desconexión de hemisferios como tratamiento de una grave epilepsia– que permiten pensar que no basta con todo el cerebro sino con una porción suficiente que asegure la identidad personal. Así, el \textit{criterio físico} plantea que la persona $P_{2}$ en $t_{2}$ es la misma persona que $P_{1}$ en $t_{1}$ si y solo si suficiente cerebro de $P_{1}$ en $t_{1}$ sobrevive en $P_{2}$ en $t_{2}$.


Según estos criterios, el cerebro es responsable del carácter, la personalidad y la memoria pero, ¿los estados mentales son estados cerebrales? ¿El cerebro es responsable de la conciencia? ¿Puede definirse el concepto de \textit{conciencia}? 

\end{document}

%%% Local Variables:
%%% mode: latex
%%% TeX-master: t
%%% TeX-output-dir: "/home/oub/ALLES/HGs/TFG-2020/build"
%%% End:

