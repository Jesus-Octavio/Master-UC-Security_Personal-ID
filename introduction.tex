\documentclass[main.tex]{subfiles}

\begin{document}
	
	\chapter*{Introducción.}
	\chaptermark{Introducción.}
	\markboth{Introducción.}{}

	Cuando vemos una cosa en un instante y espacio determinados, tenemos la certeza, sea cual sea la cosa, de que es la misma cosa que vemos y no otra que, al mismo tiempo, exista en otro lugar por semejante e indistinguible que pueda ser en todos los demás aspectos.

	Precisamente en eso consiste la identidad, en aquellas ideas que atribuimos a las cosas y que no varían en nada desde el momento en el que consideramos la existencia previa de dicha cosa. 

	La discusión filosófica sobre la identidad personal se refiere, fundamentalmente, a la identidad de una persona a través del tiempo y el espacio. Pretende dilucidar qué criterio ha de aplicarse para afirmar que una persona en un instante y espacio determinados se corresponde con la misma persona en otro instante y espacio. Consideremos la siguiente situación:

	\textit{Cuando llega la noche, para descansar de los agobios del día, me siento en la poltrona y fijo la mirada en la vieja foto en blanco y negro que cuelga de la pared del salón. En ella, veo a un niño con su madre, posando ante un paisaje marítimo. De vez en cuando me hundo en la perplejidad: ese niño soy yo. ¿Cómo es posible? Lo sé porque me lo llevan diciendo desde hace años, pero ¿qué tiene que ver ese niño conmigo? No nos parecemos físicamente y, con toda seguridad, nuestros pensamientos y actitudes ante la vida no tienen nada en común. ¿No podía yo ser, o mejor aún, haber sido otro niño? ¿Qué nos hace a ese niño de diez 10 años y a mí, de más de 20, la misma persona?}
	
	
	Cabe preguntarse: ¿existen tests que puedan aplicarse para determinar la identidad personal a través del tiempo? ¿No los usan desde hace mucho tiempo las autoridades policiales o judiciales? ¿Cuándo y por qué podremos decir que el veinteñero que descansa es, o no, la misma persona que el niño de la foto?
	
	Pero, el empeño filosófico en desentrañar el concepto de \textit{identidad personal} no nace sólo por el propio interés de aprender algo importante sobre nosotros mismos sino por la vertiente práctica: importa mucho saber en qué consiste ser la misma persona -y no el mismo ser humano u objeto físico- ya que, desde los puntos de vista ético y jurídico, resulta decisivo poder determinar las condiciones que una persona ha de cumplir para ser considerada agente responsable y, por tanto, ser objeto de la atribución de recompensas y castigos.
 
	
 	En el actual mundo tecnológico cabe preguntarse si sólo pueden atribuirse recompensas y castigos a las personas, ¿no podríamos hacer lo propio con los algoritmos? La teoría de la Inteligencia Artificial fuerte plantea un mundo de posibilidades que no podemos dejar de considerar.
	
    
	
\end{document}