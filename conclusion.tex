\providecommand{\main}{..}      % UB:28.11.2020: needed for the
                                % bibfile which is local

\documentclass[main.tex]{subfiles}

\begin{document}
%\listoftodos
\chapter*{Conclusiones y trabajo futuro}
	\chaptermark{Conclusiones y trabajo futuro}
	\markboth{Conclusiones y trabajo futuro}{}
		
Durante el desarrollo de esta memoria, han surgido diversas
cuestiones que pueden resultar interesantes de cara a un futuro
trabajo:
	
\begin{enumerate}[label=\textbullet]
		
		
		
  \item El algoritmo SLM.
		
  Hemos sido incapaces de obtener resultados adecuados en la
  resolución de la ecuación de Lane-Emden mediante el método SLM. 
  Tampoco hemos detectado errores en el código implementado en MATLAB. 
  Por tanto, sería interesante estudiar la aplicación del SLM en otros
  problemas con el fin de detectar posibles errores y comprobar su
  efectividad.
	
  \item Otros métodos numéricos para resolver la ecuación de
  Lane-Emden.
		
Tan sólo hemos considerado los algoritmos SLM y QLM
  para su resolución. 
  Si bien, son muchos los métodos históricamente aplicados para
  resolverla. 
  Una posibilidad de trabajo futuro consiste en estudiar otros métodos
  y comprobar su precisión. 
		
		
		
		
  \item El método pseudoespectral de Chebyshev y el Análisis de Fourier.
		
  Hemos optado por construir la matriz de diferenciación de Chebyshev siguiendo el método de los residuos
  ponderados, pero también es posible hacerlo mediante la {\itshape
   transformada rápida de Fourier}. 
  La clave de este desarrollo está en la equivalencia entre el
  desarrollo en serie de Chebyshev en $x\in\left[-1,1\right]$, el desarrollo en
  serie de Laurent para $z$ sobre la circunferencia unidad y el
  desarrollo en serie de Fourier en $\theta\in\mathds{R}$. 
  Por tanto, cualquier polinomio de grado $n$ podrá expresarse como
  combinación única de polinomios de Chebyshev, como desarrollo de
  Laurent, o mediante un polinomio trigonométrico y par de período
  $2\pi$. 
  Siguiendo una idea similar, una función arbitraria $f(x)$ definida
  en $[-1,1]$ tendría una función recíproca
  {\fontfamily{lmss}\selectfont f}$(z)$ sobre la circunferencia unidad
  y una función periódica $F(\theta)$ en $\mathds{R}$.
		
  Dos de las referencias principales para el desarrollo de esta
  memoria abordan este punto: 
 el libro \cite{boyd2001chebyshev} se basa en la equivalencia entre
  el desarrollo en serie de Fourier y el desarrollo en serie de Chebyshev y el manual \cite{trefethen2000spectral}  analiza el método pseudoespectral de Chebyshev desde esta doble perspectiva.
		
  Además, tales referencias abordan el empleo de la matriz de
  diferenciación de Chebyshev para resolver problemas en derivadas
  parciales, otra posible vía de trabajo.
		
		
		
		
		
		
  \item Otros métodos pseudoespectrales y matrices de diferenciación.
		
  En [\cite{alici2003pseudospectral}, Section 3.2, pp. 35-41]
  encontramos los {\itshape teoremas de Welfert}, que describen las
  entradas de una matriz de diferenciación $D^{(\ell)}$. 
  Siguiendo su desarrollo, pueden obtenerse, por ejemplo, la matriz de
  diferenciación de Legendre o emplear los polinomios de Laguerre para
  obtener el método pseudoespectral del mismo nombre, útil para
  resolver PCs en $[0, \infty)$. 
  También en [\cite{boyd2001chebyshev}, Appendix A, pp. 495-513], se
  aborda la elección de otras funciones base y se estudian los dominios de convergencia en el plano complejo.
						
				
			
  \item Profundización en la ecuación de Lane-Emden. 
		
  La interpretación física del modelo estelar es otra posible vía de
  trabajo. 
  Hemos seguido la teoría no relativista para modelar la estructura
  interna de polítropos newtonianos de manera simple gracias a la
  ecuación de Lane-Emden. 
  Resolviéndola, aparte de funciones para la densidad, presión 
  central,
  energía potencial gravitatoria e incluso temperatura de la 
  estrella en consideración, las
  soluciones permiten hallar el límite de 
  Chandrasekhar
  para una enana blanca. 
  Esta es la máxima masa posible de una enana blanca: si se supera,
  colapsará. Sin embargo, la teoría de Newton no se ajusta a objetos muy compactos y masivos. 
  Existe un límite superior para la masa de estrellas de neutrones que no puede predecirse con las ecuaciones newtonianas pero sí en el marco de la relatividad general. 
  Para las estrellas relativistas que permiten identificar modelos estables y no estables a través de la relación masa-radio, la ecuación (\ref{eqn:EHidrostatico_1}) se reemplaza por la ecuacion de Tolman-Oppenheimer-Volkoff. Haciendo un estudio similar, puede deducirse la ecuación de Lane-Emden relativista.

		
		
		
		
\end{enumerate}
\biblio
\end{document}

%%% Local Variables:
%%% mode: latex
%%% eval: (my-set-castellano)
%%% TeX-master: t
%%% TeX-output-dir: "./build"
%%% End:
